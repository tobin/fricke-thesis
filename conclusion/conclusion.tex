\CHAPTER{Conclusion}
\label{conclusion}

Over the last several decades, the state of the art of gravitational
wave detection has advanced to the point where we are likely to
discover gravitational waves with the detectors currently under
construction.

Enhanced LIGO successfully demonstrated the viability of DC readout as
a low noise interferometer readout technique.  The DC readout system
(including the output mode cleaner) in Enhanced LIGO alleviated the
problems experienced with the heterodyne readout in initial LIGO,
allowed us to increase the interferometer input power (increasing the
detectors' sensitivity), and delivered the expected shot-noise-limited
performance.  The dual Enhanced LIGO goals of both increasing the
detector sensitivity and gaining early experience with Advanced LIGO
technologies were achieved.

During Enhanced LIGO we identified OMC alignment as a particularly
important and unexpectedly subtle aspect of the OMC system, and
identified and implemented an effective alignment system.  We also
gained valuable experience in the mitigation of beam jitter coupling.

Measurements of the couplings of laser and oscillator noises reveal
that the couplings are generally improved over RF readout.  Comparison
of measured laser noise couplings with simple plane-wave models
reveals that more sophisticated models (most likely incorporating
higher-order spatial modes) are necessary to explain the measured
couplings.  The effect of spurious higher order modes will be
mitigated in Advanced LIGO through the use of geometrically stable
cavities, better optics, and improved thermal compensation; however,
to the extent that the design relies on achieving the noise couplings
predicted by a simple plane-wave model, I expect a long period of
commissioning in order to achieve it.  



