\chapter{Tables}
\label{chapter6}
\section{Abbreviations}
\singlespacing
\begin{tabular}{|l|l|}
\hline 
AF   & audio frequency \\
AM   & amplitude modulation \\
AS   & antisymmetric (output) port \\
ASC  & angular sensing and control \\
BS   & beamsplitter \\
CARM & common arm length \\
CM   & common mode \\
DARM & differential arm length; the gravitational wave readout channel \\
DC   & direct current (i.e. zero frequency) \\
ETM  & end test mass \\
FSR  & free spectral range \\
FWHM & full width, half max \\
GW   & gravitational wave \\
IFO  & interferometer \\
ISC  & interferometer sensing and control \\
ISS  & intensity stabilization servo \\
ITMX & input test mass, X arm \\
LHO  & LIGO Hanford Observatory \\
LIGO & Laser Interferometer Gravitational-wave Observatory \\
LLO  & LIGO Livingston Observatory \\
LSC  & length sensing and control \\
LSC  & LIGO Scientific Collaboration \\
MC   & mode cleaner \\
MICH & Michelson interferometer \\
OMC  & output mode cleaner \\
PD   & photodiode \\
PM   & phase modulation \\
PRC  & power recycling cavity \\
PSL  & pre-stabilized laser \\
RF   & radio frequency \\
RIN  & relative intensity noise $(\delta P/P)$ \\
SNR  & signal-to-noise ratio \\
\hline
\end{tabular}

\section{symbols}

\begin{tabular}{|l|l|}
\hline
$\nu_0$         & free spectral range \\
$f$             & frequency (Hz) \\
$f_c$           & cavity pole \\
$\mathcal{F}$   & arm cavity finesse \\
$g_\phi$         & phase gain \\
$s$             & Laplace parameter ($s=i\omega$) \\
$\omega$        & angular frequency ($\omega=2\pi f$) \\
$L$             & cavity length \\
$\nu$           & optical frequency (Hz) \\
\hline
\end{tabular}
