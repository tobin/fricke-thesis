\CHAPTER{DC readout - evaluation}
\label{chapter5}
\doublespace
\SECTION{Noise couplings - modeled and measured}

The coupling of noises from the laser source and RF oscillators to the
gravitational wave readout channel differ considerably in RF and DC
readouts.  In addition, DC readout with an OMC is generally much more
sensitive to beam motion (jitter).  These couplings are of primary
interest in designing the optical readout of a gravitational wave
detector.

\SUBSECTION{Laser frequency noise}
\SUBSECTION{Laser intensity noise}
\SUBSECTION{Oscillator amplitude noise}
%
Reduced coupling of noises from the RF oscillator is one of the motivations
for implementing DC readout. Despite not relying on the sidebands
directly, behavior of the RF oscillator is still able to sneak into
the DC readout. 

One coupling path is conversion of Oscillator AM to laser intensity
AM via the RF modulator. The sideband RIN per oscillator AM is given
by \[
2\Gamma_{0}\frac{J_{1}'(\Gamma_{0})}{J_{1}(\Gamma_{0})}=\Gamma_{0}\frac{J_{0}-J_{2}}{J_{1}}\]
(see Livingston elog 2010-07-15, {}``sideband RIN per oscillator
AM calculation,'' ).

Measured coupling is shown in figure \ref{fig:oscillator-AM-coupling-measured}.


\SUBSECTION{Oscillator phase noise}
%
Oscillator phase noise sneaks in via some dirt effect. Look at how
much better it is in DC readout!

Measured coupling of oscillator phase noise to the readout channel
is shown in figure \ref{fig:oscillator-PM-coupling-measured}.

\SUBSECTION{Beam jitter noise}
%
Beam jitter noise is perhaps the most important new noise source in
DC readout and is tied to alignment of the OMC, which has proved to
be one of the most subtle new issues. The output mode cleaner converts
motion of the input beam into variations in transmission. The resulting
variation in the transmitted light level is indistinguishable from
DARM motion.

\SUBSECTION{Electronics noise}
%
The DC readout system provides a quantum-limited readout at $\sim100$
Hz. Careful engineering ensures that we are not limited by electronics
thermal noise or $1/f$ flicker noise.

\SUBSECTION{Optical spring}
%
Detuning the arm cavities from their resonance introduces a big optical
spring. This doesn't seem to have any measurable effect.

\SUBSECTION{Nonlinearity  of the DC error signal}
%
Although we operate sufficiently far from the dark fringe that the
linear coupling of residual DARM motion to output power is dominant,
sufficiently large motion could produce second-order coupling. Fortunately,
this turns out to be totally negligible.

\SECTION{Noise budget}
\SECTION{Optical gain}
