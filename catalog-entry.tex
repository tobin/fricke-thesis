\documentclass{article}
\pagestyle{empty}
\begin{document}
\noindent 
Fricke, Tobin Thomas, B.S., University of California, Berkeley, 2003\\
M.A., University of Rochester, 2005\\
Doctor of Philosophy, Fall Commencement, 2011\\
Major: Physics\\
\underline{Homodyne Detection for Laser-Interferometric Gravitational Wave Detectors}\\
Dissertation directed by Professor Gabriela Gonz\'alez\\
Pages in dissertation, 109.  Words in abstract, 142.\\

\noindent\centerline{ABSTRACT}

Gravitational waves are ripples of space-time predicted by Einstein's
theory of General Relativity.  The Laser Interferometer
Gravitational-wave Observatory (LIGO), part of a global network of
gravitational wave detectors, seeks to detect these waves and study
their sources.

The LIGO detectors were upgraded in 2008 with the dual goals of
increasing the sensitivity (and likelihood of detection) and proving
techniques for Advanced LIGO, a major upgrade currently underway.  As
part of this upgrade, the signal extraction technique was changed from
a heterodyne scheme to a form of homodyne detection called DC readout.
The DC readout system includes a new optical filter cavity, the output
mode cleaner, which removes unwanted optical fields at the
interferometer output port.

This work describes the implementation and characterization of the new
DC readout system and output mode cleaner, including the achieved
sensitivity, noise couplings, and servo control systems.

\end{document}
