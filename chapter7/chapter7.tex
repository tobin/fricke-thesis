\CHAPTER{Appendix}
\label{chapter7}
\onehalfspace
\SECTION{The difference between PM and AM}
\label{sec:am-vs-pm}
%
Suppose we have a signal consisting of a carrier (at frequency $\omega$ and
with unit amplitude) and two sidebands, of amplitudes $a$ (lower) and $b$
(upper), separated from the carrier by a frequency $\Omega$:
%
\begin{equation}
E(t) = \left(1 + a \exp(-i \Omega t) + b \exp(i \Omega t)\right)
\exp(i \omega t)
\end{equation}
%
To find the power in this signal, we take the
modulus squared, $P = E^*E$ where * is the complex conjugate:
%
\begin{equation}
\begin{split}
P = & \left(1 + |a|^2 + |b|^2\right) \\
    & + \left(a^* + b\right) \exp(-i \Omega t) + \left(a + b^*\right) \exp(i \Omega t) \\
    & + \left(ab^*\right) \exp(-2 i \Omega t) + \left(a^*b\right) \exp(2 i \Omega t)
\end{split}
\end{equation}
%
The condition for the $1\Omega$ variation in the power to vanish is
$a=-b*$, i.e. the real parts of the amplitudes of the sidebands must
be opposite, and the imaginary parts must be equal. So we can extract
the amplitude and phase modulation indicies:
%
\begin{equation}
\begin{split}
m_{AM} &= (a + b^*)\\
m_{PM} &= (a - b^*)
\end{split}
\end{equation}

What is the condition for the $2\Omega$ signal to vanish? With just
two sidebands, it will always be present (though at second order in
the sideband amplitude). In true phase modulation, the $2\Omega$
signal is cancelled by the interaction of (the infinite number of)
higher-order sidebands. As best I can tell, there is no simple
arrangement of this cancellation other than via a magical property of
the Bessel functions.

\SECTION{Only the signal field matters}

Suppose we have two electric fields incident on a photodiode: the
signal field $A_s$ and the local oscillator field $A_{LO}$.  The power
seen by the photodiode is
$$ \left| A_s + A_{LO} \right|^2 = 
   |A_s|^2 + |A_{LO}|^2 + 2 \mathrm{ Re } {A_s}^*A_{LO}$$
In the small-signal regime, $|A_s| << |A_{LO}|$.  The signal on the
photodiode is proportional to $A_s A_{LO}$ while the shot noise is
proportional to $\sqrt{|A_s|^2+|A_{LO}|^2}\approx A_{LO}$.  
  The detected SNR is independent of the local oscillator strength.

\SECTION{Optical phase conventions}
\SECTION{The optical spring}
\SECTION{Gaussian beams}
\SECTION{Laser modes}

The eigenmodes of an optical cavity formed from spherical lenses are
the Hermite-Gauss (if the cavity has rectangular symmetry) or
Laguerre-Gauss (for axial symmetry) modes.  The amplitude distribution
at the beam waist is a Gaussian multiplied by a Hermite or Laguerre
polynomial.  These are exactly the same familes of functions as the
energy eigenstate wavefunctions of the simple harmonic oscillator in
quantum mechanics.

\SECTION{Control theory basics}

Operation of the LIGO detectors relies crucially on feedback control
systems.  In general, the response of the optical plant is very
nonlinear; in order to produce a valid readout, the plant must be held
very close to its operating point.

\SECTION{References}
\cite{Quetschke2007Complex}
