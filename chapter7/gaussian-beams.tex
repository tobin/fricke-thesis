\section{General theory of mode cleaners}

Critically-coupled resonant cavities act as optical filters, allowing
the resonant mode to be transmitted while all orthogonal modes are
highly attenuated.  These cavities can be not only
frequency-selective, but also select a single spatial mode.

Gaussian beams and higher-order modes accumulate phase slightly faster
than plane waves.  The extra phase (beyond the $\phi=kz$ expected for
a plane wave) is the Guoy phase.  It accumulates most rapidly as a
beam goes through a focus (waist). A Gaussian beam accumulates
$180^o$ as one moves from $z=-\infty$ to $z=+\infty$ through the
waist; a higher-order Hermite Gauss mode accumulates more.  The Guoy
phase shift as a function of beam axis coordinate is given by
%
\begin{equation}
\phi_g(z) = \left(1 + n + m\right) \tan^{-1} \left( \frac{z - z_0}{z_R} \right)
\label{eq:guoy-phase}
\end{equation}
%
where $z$ is the coordinate along the beam axis, $z_0$ is the location
of the waist, $z_R$ is the Rayleigh range of the beam, and $n$ and $m$
are the mode indices of the Hermite Gauss mode.

A two-mirror symmetric cavity of length $L$ with mirrors of radius of
curvature $R$ has a $g$-factor of $g = 1 - L/R$; from this we can
calculate the Rayleigh range of the mode in the cavity:

\begin{equation}
z_R = \frac{L}{2}\sqrt{\frac{1 + g}{1 - g}}
\end{equation}

\begin{align}
\phi^{cav}_n & =  \left(n+1/2\right)\left(\tan^{-1} (L/2)/z_R - \tan^{-1} (-L/2)/z_R\right) \\
            & = \left(n+1/2\right)\cos^{-1} g
\end{align}

The higher order modes have the same wavefront curvature as the lowest
order gaussian mode.  They differ only in experiencing the extra Guoy
phase.  The result is that higher order modes resonate at a slightly
different (shorter) cavity length than the fundamental mode of the
same frequency.

For resonance to occur, the total phase shift from one mirror to the
other, including both the plane wave propagation phase
$kL=\pi\nu/\nu_0$ and the Guoy phase shift, must be a multiple of
$\pi$ (so that the roundtrip phase is a multiple of $2\pi$):
%
\begin{equation}
kL - 2\left(m+n+1\right)\cos^{-1} g = \pi(q+1),
\quad q\in\mathbb{Z}
\end{equation}

For the special case of $L = 2 z_R$, the increase in the Guoy phase as
$m$ or $n$ is increased is exactly $\pi$ radians.  For this special
case--a \emph{confocal cavity}--all of the spatial modes at a given
frequency resonate simultaneously.  For a mode cleaner cavity, we want
precisely the opposite; we will carefully choose the cavity geometry
to separate higher-order spatial modes from the fundamental mode of
the laser carrier.

We can express the additional Guoy phase shift of higher order modes
as an equivalent frequency shift that would allow them to resonate at
the same cavity length as the carrier.  The effective frequency shift is
%
\begin{equation}
\Delta\nu = \frac{2}{\pi} \nu_0 \cos^{-1} g
\label{eq:hom-frequency-shift}
\end{equation}
%
where $\nu_0$ is the free spectral range.

\begin{figure}[t]
\centerline{\includegraphics[width=0.5\columnwidth]{figures/as_spot.png}}
\caption[Beam spot at anti-symmetric port]{\label{fig:as-spot}Image of
  the beam spot at the output port taken using a CCD camera. Although
  this image is saturated in the central portion, it shows the messy
  junk light surrounding the central gaussian beam.  This includes
  contributions from both the carrier and the 25 MHz sidebands.}
\end{figure}

\section{Eigenmodes}

A phase-front that returns to its original state after making a
round-trip through the optical cavity will interfere constructively
with itself, resonating in the cavity.  The set of such amplitude
distributions which are stable in the cavity and do not mix with one
another are collectively the \emph{eigenmodes} of the cavity.  The
eigenmodes of a cavity made with spherical mirrors are parametrized by
either the Hermite-Gauss or Laguerre Gauss modes.  (The same functions
also describe the energy eigenfunctions of the quantum harmonic
oscillator.)

The Hermite-Gauss modes are given by:

\begin{equation}
E_{nm}(x,y) = E_0 
\sqrt{ 
  \frac{2}{\pi w n! m! 2^n 2^m}
}
\exp \left\{-\frac{x^2+y^2}{w^2}\right\}
H_m\left( \frac{\sqrt{2}x}{w} \right)
H_n\left( \frac{\sqrt{2}y}{w} \right)
\end{equation}

\begin{figure}
\includegraphics[width=\columnwidth]{figures/hgmodes.pdf}
\caption[Hermite-Gauss TEM modes]{Hermite-Gauss modes. Positive amplitude is indicated in red and negative amplitude in blue.}
\end{figure}
