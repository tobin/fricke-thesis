\CHAPTER{The LIGO detector}
\label{chapter2}
\doublespace

\SECTION{Arms, cavities, etc}

By a convenient coincidence, a Michelson interferometer is ideally
suited to gravitational wave detection.  A suitably polarized wave
directly excites the differential mode of the Michelson, while the
Michelson simultaneously provides huge level of common mode noise
rejection.

\cite{Fritschel2001Readout}

\SECTION{Summary of changes made for Enhanced LIGO}

After the successful completion of the Initial LIGO science goals, it
was decided
\cite{Adhikari2006Enhanced,T050252,JoshSmithEnhancedAdvanced} to
attempt to further improve the detector sensitivity by aggressively
implementing a few prototype Advanced LIGO technologies.

Changes included:
\begin{itemize}
\item Fancy new laser
\item Renovation of the input optics to handle higher power operation, including new electro-optic modulators \cite{Quetschke2008ElectroOptic}
\item Installation of an output mode cleaner, supported by a prototype of the advanced LIGO in-vacuum seismic isolation system
\item New Thermal Compensation System to handle the higher power
\end{itemize}

DC readout has been implemented previously at the Caltech 40 meter
prototype \cite{Ward2008DC,RobWardThesis} and the GEO 600
detector\cite{GeoDC,Prijatelj2010,Degallaix2010Commissioning}.  The
current configuration of Virgo incorporates an output mode cleaner but
uses RF heterodyne readout\cite{Acernese2008Virgo}.

\SECTION{History of the noise during Enhanced LIGO}

\SECTION{Future directions}

