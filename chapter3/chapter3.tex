\CHAPTER{DC Readout - Theory of Operation}
\label{chapter3}
\doublespace
\SECTION{Introduction}

The basic problem is to detect optical phase.  Detectors are generally
much too slow to detect the optical field \emph{amplitude} directly (a
laser wavelength of 1064 nm corresponds to about 3 THz).  Instead,
photodetectors measure the field \emph{power}, averaged over some
interval (typically MHz).  To sense the optical phase, we turn to
interference.  By providing a second optical field--a \emph{local
  oscillator}--as a phase reference, phase perturbations of the signal
beam are converted to power fluctuations at the photodiode.  

If the local oscillator is at the same optical frequency as the signal
beam, then the photocurrent directly mimics the phase perturbations of
the signal; this is \emph{homodyne detection}, or direct conversion.
If, on the other hand, the local oscillator offset slightly in
frequency, the signal will appear as amplitude modulation on the
photodiode signal.  This is \emph{heterodyne detection}.

There is also the question of how to get the local oscillator to the
detection photodiode, and how to guarantee that it is sufficiently
stable in phase, and well-matched to the spatial distribution of the
signal beam.  One approach is to specially pipe the local oscillator
to the detection port, combine the signal and local oscillator via a
beamsplitter, and detect the two emerging beams on photodiodes, adding
the resulting signals electronically.  This scheme, balanced homodyne
detection, is depicted in \checkme{a figure}.

The problem is elegantly addressed in the current gravitational wave
detectors by having the local oscillator signal resonate in the the
interferometer, or a portion of it.

\SECTION{Interferometer Sensing and Control}

The laser light incident on the interferometer is phase-modulated at
several frequencies, producing optical sidebands on the laser carrier.
Light extracted from the interferometer at its various ports is
incident on photodiodes; the photodiode signals are demodulated to
produce signals.

\SECTION{Heterodyne detection}

In the context of coupling lasers to resonant cavities, heterodyne
detection is known as the Pound-Drever-Hall technique.  The canonical
reference is~\cite{Drever1983Laser}, while
\cite{Black2001Introduction} provides a pedagogical introduction.

\SUBSECTION{The Pound-Drever-Hall technique}

The phase modulation may be expanded in terms of sidebands using the
Jacobi-Anger expansion:\footnote{This expansion comes from the 
generating function 
$\exp\{\frac{1}{2} z \left(t - t^{-1}\right)\} = \sum_{m=-\infty}^\infty t^m J_m(z)$ with $t = \exp \{i \omega t'\}$}

\begin{equation}
\begin{array}{rcll}
\exp\{i\Gamma\cos\Omega t\} & = &\sum_{n=-\infty}^{\infty} \left(i^n\right) & J_n(\Gamma) \exp\{i n \Omega t\} \\
\exp\{i\Gamma\sin\Omega t\} & = & \sum_{n=-\infty}^{\infty} & J_n(\Gamma) \exp\{i n \Omega t\}
\end{array}
\end{equation}

Upon reflection from a cavity the phases of the carrier and the two
sidebands will be rotated.  This phase rotation converts the phase
modulation to amplitude modulation, which is observed by the
photodiodes.  Typically, the carrier is held on resonance, where there
is a very large change in reflected phase for a small change of
detuning; the sidebands, on the other hand, are typically far from
resonance and experience very little phase change.

The reflectivity of a lossless Fabry-Perot cavity is:
\begin{equation}
r_c = \frac{r_1 - r_2 \exp i2\phi}{1 - r_1 r_2 \exp i 2\phi}
\end{equation}

Suppose the carrier and upper and lower sidebands are incident upon this
cavity, and the reflection from the cavity is incident on a photodiode.
If the cavity length is $L$ and the optical frequency is $\nu$ then
$\phi = (2\pi)(L/c)\nu$.  The sideband detunings are offset from this by
$\pm(2\pi)(L/c)\Omega$.  So, as a function of detuning, we have:
%
\begin{equation}
E_r = J_0 r_c(\phi) 
+ J_1 r_c(\phi - (2\pi L/c)\Omega) 
+ J_1 r_c(\phi + (2\pi L/c)\Omega)
\end{equation}

...

Now consider a Michelson interferometer where the arms have lengths
$L_{\{x,y\}}$.  The field at the output port (at a given frequency) is:
%
\begin{equation}
E_{AS} = E_0 \frac{1}{2} \left( e^{2 i L_{x} \nu /c} + e^{2 i L_{y} \nu / c} \right) 
\end{equation}

\SECTION{Homodyne detection}

\SECTION{Comparison}

In addition to mitigating technical difficulties of RF detection,
homodyne detection confers a fundamental improvement in SNR by up to a
factor of ??. The extra noise in heterodyne detection can be
considered either a result of time dependence in the average power
leading to correlations in the shot
noise\cite{Niebauer1991Nonstationary}, or the simple fact that
demodulation introduces noises from around $2f_{mod}$, giving
an extra dose of shot noise.  A more sophisticated analysis ascribes
this noise to the two heterodyne demodulation quadratures acting as
non-commuting quantum operators\cite{Buonanno2003Quantum}.
