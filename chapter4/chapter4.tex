\CHAPTER{Output Mode Cleaner}
\label{chapter4}
\doublespace
\SECTION{Introuction}
The interferometer has many degrees of freedom (both longitudinal and
angular).  We elect only to use DC readout for the DARM degree of
freedom and continue to use heterodyne readout for (most of) the other
degrees of freedom.  Therefore we still require the sidebands to sense
these degrees of freedom, so we can't simply turn them off.  Instead
we introduce an \emph{output mode cleaner} at the output port in order
to remove them there.

\SECTION{Physical design of the cavity}

The output mode cleaner (OMC) consists of a four-mirror bowtie cavity
with a finesse of approximately 350. The cavity optics and detection
photodiodes are attached to tombstones which are bonded to a glass
slab. This entire optical bench is suspended by a double pendulum,
which in turn rests on an active seismic isolation platform, all of
which is contained within the vacuum envelope. For convenience the
chamber containing the OMC is isolated from the rest of the LIGO vacuum
envelope via a septum window, allowing the independent venting of
this chamber for easier access.

The output of the OMC is split via a 50/50 beamsplitter and directed
onto two high-quantum-efficiency InGaAs photodiodes. The two photodiode
signals allow the formation of sum and difference signals, the difference
providing a diagnostic `nullstream'.

\SECTION{Requirements}


The OMC is required to sufficiently filter the light present at the
output port such that contributions from the RF sidebands and higher-order
spatial modes become negligible. To exclude the RF sidebands, the
cavity length is chosen such that the RF sideband frequencies are
anti-resonant in the cavity, which yields minimum transmission.

\SECTION{Length control}
\SECTION{Input beam control (ASC)}
\SECTION{Automatic gain control}
\SECTION{Residual fluctuations}
\SECTION{Optical characterization}
\SUBSECTION{Mode scan}

With the interferometer controlled using the heterodyne readout, the
Output Mode Cleaner can be used as a mode analyzer cavity by varying
the cavity length by at least a free-spectral-range. Because this
range is more than the fast PZT actuator, this is accomplished by
putting a large step into the thermal actuator. These mode scans can
address questions such as:
\begin{itemize}
\item How well aligned is the OMC?
\item How well mode-matched is the OMC?
\item How much carrier power is at the output port compared to sideband
power?
\item How well balanced are the RF sidebands?
\item How much junk light is present at the output port?
\item Are there any nasty modes near the 00 mode that will sneak through?
\item The horizontal/vertical mode separation
\end{itemize}
The mode scan cannot, by itself, distinguish carrier mode light from
the arms vs carrier mode junk light.


\SUBSECTION{Scattering}


\SECTION{Interferometer lock acquisition}

Initial lock acquisition of the Enhanced LIGO interferometers is the
same as in Initial LIGO. Once the interferometer is locked using the
heterodyne readout schemes, a DARM offset is introduced to allow carrier
light to be transmitted to the output port. The Output Mode Cleaner
is then locked to this carrier light. Once the OMC is locked to the
carrier, control of DARM is transferred to the DC readout system.
After this transition, a few other changes are made to engage the
OMC alignment servoes and to put the readout electronics into low
noise mode. At this point the interferometer has reached its operation
configuration and astrophysical data-taking ({}``science mode'')
begins.


\SECTION{Beam diverter}

When the interferometer loses lock, the stored power must be dumped
somewhere. Typically, due to the presence of the power recycling mirror,
the stored power comes out the output port. This high-power transient
is sufficiently strong to burn the detection photodiodes. In order
to prevent this, one of the steering mirrors is used as a fast shutter.
It is able to zero the transmission through the OMC in approx 2 ms.
